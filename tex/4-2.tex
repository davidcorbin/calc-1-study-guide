%
% Chapter 4.2
%

\section*{4.2 Definite Integral}

\subsection*{Definition of a Definite Integral}

If \(f\) is a function defined for \(a \leq x \leq b\), we divide the interval \([a, b]\) into \(n\) subintervals of equal width \(\Delta x=\frac{(b-a)}{n}\). We let \(x_0\) (\(=a\)), \(x_1\), \(x_2\), \ldots,\(x_n\) (\(=b\)) be the endpoints of these subintervals and we let \(x_1^*, x_2^*, \ldots, x_n^*\) be any sample points in these subintervals, so \(x_i^*\) lies in the \(i\)th subinterval \([x_{i-1}, x_i]\). Then the definite integral of \(f\) from \(a\) to \(b\) is 
$$ \int_a^b f(x)dx=\lim_{n \to \infty} \sum_{i=1}^n f(x_i^*)\Delta x $$
provided that this limit exists and gives the same value for all possible choices of sample points. If it does exist, we say that \(f\) is \textbf{integrable} on \([a, b]\).

\subsection*{Precise Meaning of the Limit that Defines the Integral}

For every number \(\epsilon > 0\) there is an integer \(N\) such that
$$ \left|\int_a^b f(x)dx- \sum_{i=1}^n f(x_i^*)\Delta x\right|< \epsilon $$
for every integer \(n > N\) and for every choice of \(x_i^*\) in \([x_{i-1}, x_i]\).

\subsubsection*{Theorem}

If \(f\) is continuous on \([a, b]\), or if \(f\) has only a finite number of jump discontinuities, then \(f\) is integrable on \([a, b]\); that is, the definite integral \(\int_a^b f(x)dx\) exists.

\subsubsection*{Theorem}

If \(f\) is integrable on \([a, b]\), then
$$ \int_a^b f(x)dx=\lim_{x \to \infty} \sum_{i=1}^n f(x_i) \Delta x $$
where 
$$ \Delta x=\frac{b-a}{n} \quad \text{ and } \quad x_i=a+i \Delta x $$ 

\subsection*{Formulae for Sums of Powers of Positive Integers}

$$ \sum_{i=1}^n i=\frac{n(n+1)}{2} $$
$$ \sum_{i=1}^n i^2=\frac{n(n+1)(2n+1)}{6} $$
$$ \sum_{i=1}^n i^3= {\left [ \frac{n(n+1)}{2} \right]}^2 $$

\subsection*{Formulae for Sigma Notation}

$$ \sum_{i=1}^n c=nc \quad \text{where } c \text{ is a constant} $$
$$ \sum_{i=1}^n ca_i=c \sum_{i=1}^n a_i \quad \text{where } c \text{ is a constant} $$
$$ \sum_{i=1}^n (a_i+b_i)=\sum_{i=1}^n a_i + \sum_{i=1}^n b_i $$
$$ \sum_{i=1}^n (a_i-b_i)=\sum_{i=1}^n a_i - \sum_{i=1}^n b_i $$

\subsection*{Midpoint Rule}

$$ \int_a^b f(x)dx \approx \sum_{i=1}^{n} f(\overline{x}_i) \Delta x = \Delta x [f(\overline{x}_1)+\cdots+f(\overline{x}_n)]$$
where 
$$ \Delta x = \frac{b-a}{n} \text{ and } \overline{x}_i = \frac{1}{2}(x_{i-1} + x_i) \text{ = midpoint of } [x_{i-1},x_i] $$

\subsection*{Properties of the Definite Integral}

$$ \int_b^a f(x)dx = -\int_a^b f(x)dx $$
$$ \int_a^a f(x)dx = 0 $$
$$ \int_a^b c dx = c(b-a) \quad \text{where } c \text{ is any constant} $$
$$ \int_a^b [f(x)+g(x)] dx = \int_a^b f(x)dx + \int_a^b g(x)dx $$
$$ \int_a^b cf(x) dx = c\int_a^b f(x) dx \quad \text{where } c \text{ is any constant} $$
$$ \int_a^b [f(x)-g(x)] dx = \int_a^b f(x)dx - \int_a^b g(x)dx $$
$$ \int_a^c f(x) dx + \int_c^b f(x)dx = \int_a^b f(x)dx $$

\subsubsection*{Comparison Properties of the Definite Integral}

$$ \text{If } f(x) \geq 0 \text{ for } a \leq x \leq b \text{, then} \int_a^b f(x)dx \geq 0 $$
$$ \text{If } f(x) \geq g(x) \text{ for } a \leq x \leq b \text{, then} \int_a^b f(x)dx \geq \int_a^b g(x)dx $$
$$ \text{If } m \leq f(x) \leq M \text{ for } a \leq x \leq b \text{, then } m(b-a) \leq \int_a^b f(x)dx \leq M(b-a) $$
