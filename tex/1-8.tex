%
% Chapter 1.8
%

\section*{1.8 Continuity}

A function is continuous at a number \(a\) if 
$$\lim_{x \to a}f(x) = f(a)$$\\
A function \(f\) is continuous from the right at a number \(a\) if 
$$\lim_{x \to a^+}f(x)=f(a)$$
and \(f\) is continuous from the left at \(a\) of 
$$\lim_{x \to a^-}f(x)=f(a)$$\\
A function \(f\) is continuous on an interval if it is continuous at every number in the interval.
\\\\
If \(f\) and \(g\) are continuous at \(a\) and if \(c\) is a constant, then the following functions are also continuous at \(a\).
\begin{enumerate}
    \item \(f + g\)
    \item \(f - g\)
    \item \(cf\)
    \item \(fg\)
    \item \(\frac{f}{g}\) if \(g(a) \neq 0\)
\end{enumerate}
Any polynomial, rational function, root function, or trigonometric function is continuous anywhere on its domain.

\subsection*{Intermediate Value Theorem}

Suppose that \(f\) is continuous on the closed interval \([a, b]\) and let \(N\) be any number between \(f(a)\) and \(f(b)\), where \(f(a) \neq f(b)\). Then there exists a number \(c\) in \((a, b)\) such that \(f(c)\) = \(N\).
