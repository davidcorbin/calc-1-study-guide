%
% Chapter 3.1
%

\section*{3.1 Max and Min Values}

\subsection*{Absolute Extrema}

Let \(c\) be a number in the domain \(D\) of the function \(f\). Then \(f(x)\) is the 
\begin{itemize}
    \item \textbf{absolute maximum} value of \(f\) on \(D\) if \(f(c) \geq f(x)\) for all \(x\) in \(D\).
    \item \textbf{absolute minimum} value of \(f\) on \(D\) if \(f(c) \leq f(x)\) for all \(x\) in \(D\).
\end{itemize}

\subsection*{Relative Extrema}

The number \(f(c)\) is a 
\begin{itemize}
    \item \textbf{local maximum} value of \(f\) if \(f(c) \geq f(x)\) when \(x\) is near \(c\).
    \item \textbf{local minimum} value of \(f\) if \(f(c) \leq f(x)\) when \(x\) is near \(c\).
\end{itemize}

\subsection*{Extreme Value Theorem}

If \(f\) is continuous on a closed interval \([a, b]\), then \(f\) attains an absolute maximum value \(f(x)\) and an absolute minimum value \(f(d)\) at some numbers \(c\) and \(d\) in \([a, b]\).

\subsection*{Fermat's Theorem}

If \(f\) has a local maximum or minimum at \(c\), and if \(f'(x)\) exists, then \(f'(c)=0\).
\\\\\\
A \textbf{critical number} of a function \(f\) is a number \(c\) in the domain of \(f\) such that either \(f'(c)=0\) or \(f'(c)\) does not exist. If \(f\) has a local maximum or a minimum at \(c\), then \(c\) is a critical number of \(f\).

\subsection*{Closed Interval Method}

To find the absolute maximum and minimum values of a continuous function \(f\) on a closed interval \([a, b]\):
\begin{enumerate}
    \item Find the values of \(f\) at the critical values of \(f\) in \((a, b)\).
    \item Find the values of \(f\) at the endpoints of the interval.
    \item The largest of the values is the absolute maximum; the smallest of these values is the absolute minimum.
\end{enumerate}
