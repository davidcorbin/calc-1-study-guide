%
% Chapter 3.4
%

\section*{3.4 Limits at Infinity}

Let \(f\) be a function defined on some interval \((a, \infty)\). Then
$$ \lim_{x \to \infty}f(x) = L $$
means that the values of \(f(x)\) can be made arbitrarily close to \(L\) by requiring \(x\) to be sufficiently large.
\\\\
Let \(f\) be a function defined on some interval \((-\infty, a)\). Then
$$ \lim_{x \to -\infty}f(x) = L $$
means that the values of \(f(x)\) can be made arbitrarily close to \(L\) by requiring \(x\) to be sufficiently large negative.

\subsection*{Horizontal Asymptotes}

The line \(y=L\) is called a \textbf{horizontal asymptote} of the curve \(y=f(x)\) if either
$$\lim_{x \to \infty}f(x)=L \quad \text{or} \quad \lim_{x \to -\infty}f(x)=L$$

\subsubsection*{Theorem}

If \(r>0\) is a rational number, then
$$ \lim_{x \to \infty}\frac{1}{x^r}=0 $$
If \(r>0\) is a rational number such that \(x^r\) is defined for all \(x\), then
$$ \lim_{x \to -\infty}\frac{1}{x^r}=0 $$

\subsection*{Definition of a Limit at Positive Infinity}

Let \(f\) be a function defined on some interval \((a, \infty)\). Then
$$ \lim_{x \to \infty}f(x)=L$$
means that for every \(\epsilon>0\) there is a corresponding number \(N\) such that 
$$ \text{if} \quad x>N \quad \text{then} \quad \left| f(x)-L \right| <\epsilon $$ 

\subsection*{Definition of a Limit at Negative Infinity}

Let \(f\) be a function defined on some interval \((-\infty, a)\). Then
$$ \lim_{x \to -\infty}f(x)=L$$
means that for every \(\epsilon>0\) there is a corresponding number \(N\) such that 
$$ \text{if} \quad x<N \quad \text{then} \quad \left| f(x)-L \right| <\epsilon$$ 

\subsection*{Definition of an Infinite Limit at Infinity}

Let \(f\) be a function defined on some interval \((a, \infty)\). Then
$$ \lim_{x \to \infty}f(x)=\infty$$
means that for every positive number \(M\) there is a corresponding positive number \(N\) such that 
$$ \text{if} \quad x>N \quad \text{then} \quad f(x)>M$$ 

