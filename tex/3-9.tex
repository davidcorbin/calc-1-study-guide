%
% Chapter 3.9
%

\section*{3.9 Antiderivatives}

A function \(F\) is called an \textbf{antiderivative} of \(f\) on an interval \(I\) if \(F'(x)=f(x)\) for all \(x\) in \(I\).

\subsubsection*{Theorem}

If \(F\) is an antiderivative of \(f\) on an interval \(I\), then the most general antiderivative of \(f\) on \(I\) is 
$$ F(x) + C $$
where \(C\) is an arbitrary constant.

\subsection*{General Antiderivative Formulae}

$$ \text{The antiderivative of } \quad cf(x) \quad \text{ is } \quad cF(x) + C. $$
$$ \text{The antiderivative of } \quad f(x)+g(x) \quad \text{ is } \quad F(x) + G(x) + C. $$
$$ \text{The antiderivative of } \quad x^{n} (n \neq -1) \quad \text{ is } \quad \frac{x^{n+1}}{n+1} + C. $$
$$ \text{The antiderivative of } \quad cos(x) \quad \text{ is } \quad sin(x) + C. $$
$$ \text{The antiderivative of } \quad sin(x) \quad \text{ is } \quad -cos(x) + C. $$
$$ \text{The antiderivative of } \quad sec^2(x) \quad \text{ is } \quad tan(x) + C. $$
$$ \text{The antiderivative of } \quad sec(x)tan(x) \quad \text{ is } \quad sec(x) + C. $$
\\
An equation that involves the derivatives of a function is called a \textbf{differential equation}.
