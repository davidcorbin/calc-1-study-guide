%
% Chapter 2.9
%

\section*{2.9 Linear Approximation and Differentials}

\textbf{Linear approximation} (tangent line approximation) is a way of finding the slope of a curve using the tangent line at point \((a, f(a))\) as an approximation to the curve \(y=f(x)\) when \(x\) is near \(a\).\\\\
The linear function whose graph is the tangent line of \(f(x)\) at \((a, f(a))\) is called the \textbf{linearization} of \(f\) at \(a\).

$$ L(x) = f(a) + f'(a)(x-a) $$

\subsection*{Differentials}

If \(y=f(x)\) is a differentiable function, then differential \(dx\) is an independent variable. The differential \(dy\) is defined in terms of \(dx\) by the equation
$$ dy=f'(x)dx $$
