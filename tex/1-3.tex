%
% Chapter 1.3
%

\section{More advanced functions}

Vertical and Horizontal Shifts\\

Suppose \(c>0\). Then
\begin{center}
\(y=f(x)+c\) is the graph of \(y=f(x)\) shifted \(c\) units upward.\\
\(y=f(x)-c\) is the graph of \(y=f(x)\) shifted \(c\) units downward.\\
\(y=f(x+c)\) is the graph of \(y=f(x)\) shifted \(c\) units to the left.\\
\(y=f(x-c)\) is the graph of \(y=f(x)\) shifted \(c\) units to the right.\\
\end{center}
Vertical and Horizontal Stretching and Reflecting\\

Suppose \(c>1\). Then
\begin{center}
\(y=cf(x)\) is the graph of \(y=f(x)\) stretched vertically by a factor of \(c\).\\
\(y=(1/c)f(x)\) is the graph of \(y=f(x)\) shrunk vertically by a factor of \(c\).\\
\(y=f(cx)\) is the graph of \(y=f(x)\) shrunk horizontally by a factor of \(c\).\\
\(y=f(x/c)\) is the graph of \(y=f(x)\) stretched horizontally by a factor of \(c\).\\
\(y=-f(x)\) is the graph of \(y=f(x)\) reflected about the \(x\)-axis.\\
\(y=f(-x)\) is the graph of \(y=f(x)\) relected about the \(y\)-axis.
\end{center}
