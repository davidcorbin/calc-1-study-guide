%
% Chapter 1.6
%

\section*{1.6 Limit Laws}

Suppose that \(c\) is the constant and the limits 
$$\lim_{x-a}f(x) \quad \text{ and } \quad \lim_{x \to a}g(x)$$
exist. Then
$$\textbf{1. }\lim_{x \to a}[f(x)+g(x)]=\lim_{x \to a}f(x) + \lim_{x \to a}g(x)$$
$$\textbf{2. }\lim_{x \to a}[f(x)-g(x)]=\lim_{x \to a}f(x) - \lim_{x \to a}g(x)$$
$$\textbf{3. }\lim_{x \to a}[cf(x)]=c\lim_{x \to a}f(x)$$
$$\textbf{4. }\lim_{x \to a}[f(x)g(x)]=\lim_{x \to a}f(x) \times \lim_{x \to a}g(x)$$
$$\textbf{5. }\lim_{x \to a}[\frac{f(x)}{g(x)}]=\frac{\lim_{x \to a}f(x)}{\lim_{x \to a}g(x)} \text{ if } \lim_{x \to a}g(x) \neq 0 $$
$$\textbf{6. }\lim_{x \to a}[f(x)]^n=[\lim_{x \to a}f(x)]^n \text{ where } n \text{ is a postitive integer.}$$
$$\textbf{7. }\lim_{x \to a}c=c$$
$$\textbf{8. }\lim_{x \to a}x=a$$
$$\textbf{9. }\lim_{x \to a}x^n=a^n \text{ where } n \text{ is a positive integer}$$
$$\textbf{10. }\lim_{x \to a}\sqrt[n]x=\sqrt[n]a \text{ where } n \text{ is a positive integer}$$
$$\textbf{11. }\lim_{x \to a}\sqrt[n]{f(x)}=\sqrt[n]{\lim_{x \to a}f(x)} \text{ where } n \text{ is a positive integer}$$

\begin{enumerate}
    \item The limit of the sums is the sum of the limits.
    \item The limit of the difference is the difference of the limits.
    \item The limit of a constant times a function is the constant times the limit of the function.
    \item The limit of the product is the product of the limits.
    \item The limit of the quotient is the quotient of the limits (provided that the limit of the denominator is not 0).
\end{enumerate}

\subsection*{Direct Substitution Property}

If \(f\) is a polynomial or a rational function and \(a\) is in the domain of \(f\), then 
$$\lim_{x \to a}f(x)=f(a)$$

\subsection*{Sequeeze Theorem}

If \(f(x) \leq g(x) \leq h(x)\) when x is near \(a\) (except at \(a\)) and 
$$\lim_{x \to a}f(x)=\lim_{x \to a}h(x) = L$$
then
$$\lim_{x \to a}g(x) = L$$
