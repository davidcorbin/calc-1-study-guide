%
% Chapter 2.1
%

\section*{2.1 Derivative and Rates of Change}

\subsection*{Tangents}

The \textbf{tangent line} to the curve \(y=f(x)\) at the point \(P(a, f(a))\) is the line through \(P\) with slope 
$$m=\lim_{x \to a}\frac{f(x)-f(a)}{x-a} \Leftrightarrow m=\lim_{h \to 0}\frac{f(a+h)-f(a)}{h}$$
as long as the limit exists.

\subsection*{Velocities}

$$\text{average velocity}=\frac{\text{displacement}}{\text{time}}=\frac{f(a+h)-f(a)}{h}$$\\
Instantaneous velocity at time \(t = a\) is defined as 
$$v(a) = \lim_{h \to 0}\frac{f(a+h)-f(a)}{h}$$

\subsection*{Derivatives}

The derivative of a function \(f\) at a number \(a\), denoted by \(f'(a)\), is 
$$f'(a)=\lim_{h \to 0}\frac{f(a+h)-f(a)}{h}$$
if the limit exists.

\subsection*{Rates of Change}

The average rate of change of \(y\) with respect to \(x\) over the interval \([x_1, x_2]\) is the difference quotient
$$\frac{\Delta x}{\Delta y}=\frac{f(x_2)-f(x_1)}{x_2-x_1}$$
\\
The instantaneous rate of change of \(y\) with respect to \(x\) is 
$$\lim_{\Delta x \to 0}\frac{\Delta y}{\Delta x}=\lim_{x_2 \to x_1}\frac{f(x_2)-f(x_1)}{x_2-x_1}$$
The derivative \(f'(a)\) is the instantaneous rate of change of \(y=f(x)\) with respect to \(x\) when \(x=a\).

