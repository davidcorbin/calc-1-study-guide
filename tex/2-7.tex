%
% Chapter 2.7
%

\section*{2.7 Rates of Change Problems}

The averate rate of change of \(y\) with respect to \(x\) over the interval \([x_1, x_2]\) is
$$ \frac{\Delta y}{\Delta x} = \frac{f(x_2)-f(x_1)}{x_2-x_1} $$
Its limit as \(\Delta x \to 0\) is the derivative \(f'(x)\), which is the instantaneous rate of change of \(y\) with respect to \(x\) written as
$$ \frac{dy}{dx} = \lim_{\Delta x \to 0}\frac{\Delta y}{\Delta x} $$

\subsection*{Physics}

If \(s = f(t)\) is the position function of a particle that is moving in a straight line, then \(\frac{\Delta s}{\Delta t}\) represents the \textbf{average velocity over a time period} \(\Delta t\).\\
\(v=\frac{ds}{dt}\) represents the instantaneous velocity (the rate of change of displacement with respect to time).\\
\(a=\frac{dv}{dt}\) represents the instantaneous acceleration (the rate of change of velocity with respect to time).
$$ a(t)=v'(t)=s''(t) $$

\subsection*{Chemistry}

Consider the reaction where \(A + B \to C\). The \textbf{instantaneous rate of reaction} is obtained by taking the limit of the averate rate of reaction as the time interval \(\Delta t\) approaches 0.
$$ \text{rate of reaction} = \lim_{\Delta t \to 0}\frac{\Delta [C]}{\Delta t} = \frac{d[C]}{dt} $$

\subsection*{Biology}

Let \(n=f(t)\) be the number on individuals in an animal or plant population at time \(t\).
$$ \text{average rate of growth} = \frac{\Delta n}{\Delta t} = \frac{f(t_2)-f(t_1)}{t_2-t_1} $$
The \textbf{instantaneous rate of growth} is obtained from the average rate of growth by letting the time period \(\Delta t\) approach 0:
$$ \text{growth rate} = \lim_{\Delta t \to 0}\frac{\Delta n}{\Delta t} = \frac{dn}{dt} $$
