%
% Chapter 1.5
%

\section*{1.5 Limit of a Function}

\subsection*{Limit}

We can make the values of \(f(x)\) arbitrarily close to \(L\) by restricting \(x\) to be sufficiently close to \(a\) but never equal to \(a\).
$$\lim_{x \to a}f(x)=L \text{ means that } f(x) \to L \text{ as } x \to a$$

\subsection*{One-sided Limit}

$$\lim_{x \to a^-}f(x)=L$$
means that the limit of \(f(x)\) as \(x\) approaches \(a\) from the left is equal to \(L\) if we can make the values of \(f(x)\) arbitrarily close to \(L\) by taking \(x\) to be sufficiently close to \(a\) with \(x\) less than \(a\).

$$\lim_{x \to a}f(x)=L \quad \text{if} \quad \lim_{x \to a^-}f(x)=L \quad \text{and} \quad \lim_{x \to a^+}f(x)=L$$

\subsection*{Infinite Limits}

Let \(f\) be a function defined on both side of \(a\), except possibly at \(a\) itself. Then 
$$\lim_{x \to a}f(x)=\infty$$
means that the values of \(f(x)\) can be made arbitrarily large by taking \(x\) sufficiently close to \(a\), but not equal to \(a\).
\\\\
Let \(f\) be a function defined on both side of \(a\), except possibly at \(a\) itself. Then 
$$\lim_{x \to a}f(x)=-\infty$$
means that the values of \(f(x)\) can be made arbitrarily large negative by taking \(x\) sufficiently close to \(a\), but not equal to \(a\).

\subsection*{Vertical Asymptotes}

The line \(x=a\) is called \textbf{vertical asymptote} of the curve \(y=f(x)\) if at least one of the following statements is true:

$$\lim_{x \to a}f(x)=\infty \quad \lim_{x \to a^-}f(x)=\infty \quad \lim_{x \to a^+}f(x)=\infty$$
$$\lim_{x \to a}f(x)=-\infty \quad \lim_{x \to a^-}f(x)=-\infty \quad \lim_{x \to a^+}f(x)=-\infty$$

