%
% Chapter 5.4
%

\section*{5.4 Work}

The \textbf{force} \(F\) on an object (in the same direction) is given by Newton's Second Law of Motion as the product of its mass \(m\) and its acceleration \(a\):
\[ \text{force = mass} \times \text{acceleration} \quad \Leftrightarrow \quad F = ma = m \frac{d^2s}{dt^2} \]
For the case of constant acceleration, the force \(F\) is also constant and the work done is defined to be the product of the force \(F\) and the distance \(d\) that the object moves:
\[ \text{work = force} \times \text{distance} \quad \Leftrightarrow \quad W = Fd \]

\subsubsection*{Work Done Moving an Object from \(A\) to \(B\)}

\[ W = \lim_{n \to \infty} \sum_{i=1}^n f(x_1^*) \Delta x = \int_a^b f(x)dx \]

\subsubsection*{Hooke's Law}

The force required to maintain a spring stretched \(x\) units beyond its natural length is proportional to \(x\):
\[f(x)=kx\]
where \(k\) is a positive constant called the \textbf{spring constant}. Hooke's law holds as long as \(x\) is not too large.
