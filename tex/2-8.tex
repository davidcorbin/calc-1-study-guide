%
% Chapter 2.8
%

\section*{2.8 Related Rates}

\subsubsection*{Example}

Air is being pumped into a spherical balloon so that its volume increases at a rate of 100 cm\(^3\)/s. How fast is the radius of the balloon increasing when the diameter is 50 cm?
\\\\
Solution:
\\\\
Identify given infomation and the unknown value.
$$ \frac{dV}{dt} = 100 \text{ cm}^3/\text{s} $$
$$ \frac{dr}{dt} =\text{? when } r = 25 \text{ cm} $$
Relate \(V\) and \(r\) by a formula.
$$ V=\frac{4}{3} \pi r^3 $$
Differentiate both sides of the equation with respect to \(t\).
$$ \frac{dV}{dt}=\frac{dV}{dr}\frac{dr}{dt}=4 \pi r^2 \frac{dr}{dt} $$
Solve for the unknown value.
$$ \frac{dr}{dt} = \frac{1}{4 \pi r^2} \frac{dV}{dt} $$
If we put \(r=25\) and \(\frac{dV}{dt}=100\) in the equation, we get
$$ \frac{dr}{dt} = \frac{1}{4 \pi (25)^2}100=\frac{1}{25 \pi} \text{ cm/s} $$ 
