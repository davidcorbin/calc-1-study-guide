%
% Chapter 1.7
%

\section*{1.7 Definition of a Limit}

\subsection*{Formal Definition of a Limit}

Let \(f\) be a function defined on some open interval that contains the number \(a\), except possibly \(a\) itself. Then we say that the limit of \(f(x)\) as \(x\) approaches \(a\) is \(L\) and we write
$$\lim_{x \to a}f(x)=L$$
if for every number \(\epsilon > 0\) there is a number \(\delta > 0\) such that
$$\text{if} \quad 0< \left| x-a \right| <\delta \quad \text{then} \quad \left| f(x)-L \right| <\epsilon$$

\subsection*{Formal Definition of a Left-Hand Limit}

$$\lim_{x \to a^-}f(x)=L$$
if for every number \(\epsilon > 0\) there is a number \(\delta > 0\) such that 
$$\text{if} \quad a-\delta<x<a \quad \text{then} \quad \left| f(x)-L \right| <\epsilon$$ 

\subsection*{Formal Definition of a Right-Hand Limit}

$$\lim_{x \to a^+}f(x)=L$$
if for every number \(\epsilon > 0\) there is a number \(\delta > 0\) such that 
$$\text{if} \quad a<x<a+\delta \quad \text{then} \quad \left| f(x)-L \right| <\epsilon$$ 

\subsection*{Formal Definiton of an Infinite Limit}

Let \(f\) be a function defined on some open interval that contains the number \(a\), except \(a\) itself. Then
$$\lim_{x \to a}f(x)=\infty$$

means that for every positive number \(M\) there is a positive number \(\delta\) such that 
$$\text{if} \quad 0< \left| x-a \right| <\delta \quad \text{then} \quad f(x)>M$$


\subsection*{Formal Definiton of a Negative Infinite Limit}

Let \(f\) be a function defined on some open interval that contains the number \(a\), except \(a\) itself. Then
$$\lim_{x \to a}f(x)=-\infty$$
means that for every negative number \(N\) there is a positive number \(\delta\) such that 
$$\text{if} \quad 0< \left| x-a \right| <\delta \quad \text{then} \quad f(x)>N$$

